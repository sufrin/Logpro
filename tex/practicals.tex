\documentclass{article}
\def\LOGPROVERSION{1.44}

\usepackage[a4paper]{papersize}
\usepackage{verbatim}
%\usepackage{RCS}
\usepackage{alltt}
\usepackage{xspace}
%\RCS $Id: practicals.tex,v 2.0 2000/02/07 16:46:23 sufrin Exp sufrin $
\title{I(2) Programming Languages\\Logic Programming Laboratories}
\author{Bernard Sufrin}
\date{Version of February 7, 2002}
%
%
\parindent=0pt
\parskip=\medskipamount
\flushbottom
%
%
\def\LOGPRO{{\ttfamily\slshape LogPro}\xspace}
\newenvironment{smalltt}{\begin{small}\begin{alltt}}{\end{alltt}\end{small}}
\def\rmbox#1{\mbox{{\rm #1}}}
%
%
\begin{document}
\maketitle
\begin{abstract}
\noindent This document describes the {\it complete} set of three Labs for the
Programming Languages course. The first two labs are identical to those
in the earlier revision of this document. 

\noindent Changes:
\begin{enumerate}
\item  The syntax of \LOGPRO version 1.44 has been extended to accomodate ``fat brackets''. 
\item  \LOGPRO  1.44 has different primary and secondary prompts ({\tt --} and {\tt >>} respectively)
\end{enumerate}
\end{abstract}

\section{First Lab}
This Lab is designed to get you acquainted with the \LOGPRO system and 
with Prolog.

\begin{enumerate}
\item Read Appendix \ref{Manual} to find out how to run the \LOGPRO system.

\item Define {\tt append} in the usual way, and 
\begin{smalltt}
        isrot(XS, YS) :- append(LS, RS, XS), append(RS, LS, YS).
\end{smalltt}

Now find the answers to the query {\tt isrot(1:2:3:nil, YS)}?

What happens if you make the query  {\tt isrot(XS, 1:2:3:nil)}?
Can you explain why?\footnote{You may not be able to answer this
part of the question fully until we have explained the Prolog
execution model more fully in lectures, but if you turn on
tracing in the \LOGPRO interpreter you may be able to describe what is
happening in a little more detail, and that will be
enough for the moment.}


\item In the lectures we partly defined the relation $is$ to implement
basic arithmetic notation.
\begin{smalltt}
        #infix1 is
        #infix4 +
        #infix5 *
        V is V   :- num(V).
        V is E+F :- Ev is E, Fv is F, sum(Ev, Fv, V). 
        V is E*F :- Ev is E, Fv is F, prod(Ev, Fv, V). 
\end{smalltt}
complete the definition of this relation so that it includes division, subtraction, and 
remainder. 

\item Investigate, and explain,\footnote{See the previous footnote.} what happens when you try to use {\tt is}
``backwards'', to solve a query such as
\begin{smalltt}
        3 is V+1?
\end{smalltt}


\item Change the definition of {\tt is} so that it works for strings
as well as numbers. Overload (+) to signify catenation as well as addition, 
and use \verb.-. and \verb./. for suffix and prefix removal.  
\begin{verbatim}
        "foobar" is "foo" + "bar"
        "foo"    is "foobar" - "bar"
        "bar"    is "foobar" / "foo"
\end{verbatim}
Avoid repeated evaluation of the subexpressions of composite
expressions if you can. 
        
\begin{comment}
        #infix1 is
        #infix4 +
        #infix5 *
        V is V   :- num(V).
        V is V   :- str(V).
        V is E+F :- Ev is E, Fv is F, evplus(Ev, Fv, V). 
        V is E*F :- Ev is E, Fv is F, prod(Ev, Fv, V). 
        
        evplus(V1, V2, V) :- num(V1), num(V2), sum(V1, V2, V).
        evplus(V1, V2, V) :- str(V1), str(V2), cat(V1, V2, V).
\end{comment}


\end{enumerate}

\section{Second Lab}
A binary tree takes one of the forms

\begin{center}\begin{tabular}{ll}\tt
        leaf(\(E\))             &\rmbox{where \(E\) is a term}
        \\
        \(T\sb1\) ** \(T\sb2\)   &\rmbox{where \(T\sb1\), \(T\sb2\) are binary trees}
\end{tabular}
\end{center}
\begin{enumerate}
\item (``Cat-elimination'')
\begin{enumerate}
\item Define a predicate {\tt flatten/2} such that $flatten(T, L)$ holds 
if $L$ is the list formed by the elements at the leaves of the tree $Y$.
For example
\begin{smalltt}
        flatten((leaf(1)**leaf(2))**(leaf(3)**leaf(4)), 1:2:3:4:nil)?
\end{smalltt}
Use the predicate {\tt append/3} to form the list, and investigate 
(using the directive {\tt \#count on}) how
many inferences are needed in the best and worst cases when the result list 
is of length 4, 8, 12, 16, and 20. 

\item Define a predicate {\tt flat/2} with the same specification
as {\tt flatten}, but use open-lists and open-list-append to 
form the answer list. Investigate the number of inferences needed in the
best and worst cases as in the previous exercise.

\end{enumerate}
What do you conclude from parts (a) and (b) above?

\item Can either {\tt flatten} or {\tt flat} be used ``backwards'' -- 
in other words to form a tree from a list? 

\item Define a predicate {\tt formtree/2} such that $formtree(L, T)$ holds
if $T$ is a balanced binary tree formed from the elements of the list $L$, 
and such that $formtree(L, T)$ holds whenever $flat(T, L)$ holds.

\item Can {\tt formtree/2} be used backwards -- in other words to form a list from a tree?

\end{enumerate}

\section{Third Lab} 

In this Lab  we investigate the use of Prolog to prototype the
operational semantics of the Plain language developed in the lectures. With a
modicum of advance preparation the two parts can be taken at one sitting.

We take for granted the following \LOGPRO definitions of
the equality and inequality relations
\begin{smalltt}
        #infix3 "=" 
        #infix3 "/=" 
        X=X         :- .
        X/=Y        :- not X=Y.   
\end{smalltt}
\def\dom{\mathop{\mathbf{dom}}}

We will represent finite mappings by terms of the form $\{\}$ 
(for the empty mapping), and $U\mapsto V ~~\&~~ M$ for the mapping 
written $M\oplus\{U\mapsto V\}$ in the lecture notes.

\subsection*{Part 1} 
The following clauses define the relation $lookup(X, MAP, R)$
which holds between a term $MAP$ (representing a finite mapping $map$), 
and terms $X$ and $R$  when $X\in \dom map \land R=map(X)$.\footnote{We have deliberately
left out the clause defining $lookup$ in an empty mapping so that we
can explore the possibility of having ``stuck'' configurations.}
\begin{smalltt}
        #infix6 "|->"
        #infix5 "&"
        lookup(X, U|->V & MAP, V)      :- X=U.
        lookup(X, U|->V & MAP, R)      :- X/=U, lookup(X, MAP, R).
\end{smalltt}



\subsubsection*{Task} Define a relation $update(MAP, X\mapsto Y, MAP')$ which holds 
between terms ($MAP, MAP'$) representing mappings ($map, map'$) when
$map'=map\oplus\{X\mapsto Y\}$. In addition: if $X$ is already in the domain of
$map$ then {\em the representation of $map'$ should be no larger 
than that of $map$} (in other words should contain no more terms). For example
\begin{smalltt}
        update(                  \{\}, a|->2, a|->2 & \{\})
        update(          a|->1 & \{\}, a|->2, a|->2 & \{\})
        update(b |-> 4 & a|->1 & \{\}, a|->2, b |-> 4 & a|->2 & \{\})
\end{smalltt}


The additional requirement is present to ensure that while
the operational semantics of a program are simulated, the term
representing a store contains no more than a single value for
each variable in the store.\footnote{This is essential if humans
are to be able to make sense of the output of a simulation.}



\subsection*{Part 2} 

The inference rules of an operational semantics can often
be modelled by Prolog clauses in which the consequent of the
rule appears as the head of the clause, and the antecedents
as the right hand side. This works satisfactorily only if
the side conditions of the rules are encoded as predicates
and are used as guards ({\it i.e.\/} early) on the right hand
side.\footnote{The notational conventions we use to present
the rules in the lecture notes must also be respected by the
Prolog program, and the best way of doing so is to treat them
as additional side-conditions.}


Here, for example, is an encoding of a few of the Plain rules
as a \LOGPRO program. 

\newpage
We start with syntactic declarations
\begin{smalltt}
        #infix0 "-|>"
        #infix3 ";"
        #infix4 ":="
        #infix7 "+"
        #infix7 "-"
\end{smalltt}


Next we define predicates which capture some of the side conditions.
\begin{smalltt}
        id(I)                :- atom(I), I/=skip.
        con(E)               :- num(E).
\end{smalltt}

Our configurations are encoded as \LOGPRO terms in one of the forms\footnote{{\bf NB: the ``fat bracket'' notation is only available in
\LOGPRO versions 1.44 and later. If you don't presently have access to this,
you may encode configurations as {\tt Program | Store} after declaring 
{\tt \#infix0 |}}}
\begin{smalltt}
        [| Program    |] Store  
        [| Expression |] Store  
\end{smalltt}

and our judgements as relations of the form
\begin{smalltt}
        CONFIGURATION -|> CONFIGURATION'  
\end{smalltt}

A configuration is in normal form if its program component is {\tt skip}. 
\begin{smalltt}
        normalForm([| skip |] S) :- .
\end{smalltt}

We want to show the names of the rules which are used during the execution of
a program, so we define the predicate {\tt rule} and 
end each rule by calling it.
\begin{smalltt}
        #notfix rule
        rule S :- show(" (by", S, ")").
\end{smalltt}

The first few rules are encoded as follows\footnote{The rule for identifiers is
not quite the same as that given in the levcture notes, because $lookup$
fails when presented with variables which not already in the store. This is deliberate -- we want to
be able to produce a ``stuck'' configuration easily.}
\begin{smalltt}
    [|  I       |] S        -|>    [|  V     |] S   :- id(I),  
                                                       lookup(I, S, V),
                                                       rule "var".
                                                      
    [|  I:=E    |] S        -|>    [|  skip  |] S'  :- id(I), 
                                                       con(E), 
                                                       update(S, I|->E, S'),
                                                       rule "assign.2".

    [|  I:=E    |] S        -|>    [|  I:=E' |] S   :- id(I), not con(E), 
                                                       [| E |] S -|> [| E' |] S,
                                                       rule "assign.1".
                                                       
    [|  P; Q    |] S        -|>    [|  P'; Q |] S'  :- P/=skip,
                                                       [| P |] S -|> [| P' |] S',
                                                       rule "seq.1".

    [|  skip; Q |] S        -|>    [|  Q     |] S   :- rule "seq.2".
\end{smalltt}

The {\tt -|>} relation describes only a single step in the (top-level)
execution of a program, although this step may involve the use of
several structural inference rules in addition to any computation rule
it uses. 

For example:
\begin{smalltt}
        [| x := 3 ; y := 4 |] \{\}   -|>  CONFIG' ?
         (by assign.2 )
         (by seq.1 )
        CONFIG' = [| skip ; y := 4 |] x |-> 3 & \{\}  
\end{smalltt}

The complete program execution relation {\tt -|>* } is encoded as 
follows:\footnote{({\it For the theoretically inclined:\/}) Notice that {\tt -|>*} computes a subrelation of
the transitive closure of {\tt -|>}.
In fact it is only the ``stuck'' configurations that are not actually
reached by execution of {\tt -|>*}, even though 
the last {\tt show} it executes shows a configuration which is not 
in the domain of {\tt -|>}.}
\begin{smalltt}
    #infix0 "-|>*"
    CONFIG  -|>* CONFIG    :- normalForm(CONFIG),
                              show(CONFIG, "  finished.").           

    CONFIG  -|>* CONFIG''  :- not(normalForm(CONFIG)),
                              show(CONFIG, "  -|>"),
                              CONFIG  -|> CONFIG',
                              CONFIG' -|>* CONFIG''.
\end{smalltt}

Notice that we take pains to show the configuration at each stage, and to inform the
user when a program has finished. 
For example
\begin{alltt}
    ...
    [| x := 3 ; y := x |] \{\}   -|>
     (by assign.2 )
     (by seq.1 )
    [| skip ; y := x |] x |-> 3 & \{\}   -|>
     (by seq.2 )
    [| y := x |] x |-> 3 & \{\}   -|>
     (by var )
     (by assign.1 )
    [| y := 3 |] x |-> 3 & \{\}   -|>
     (by assign.2 )
    [| skip |] x |-> 3 & y |-> 3 & \{\}   finished.
\end{alltt}

If the program is just stuck,\footnote{For the moment the only way this is
possible is if a variable is referenced for which there is not yet a value in the
store.} the ``finished'' epithet is never applied. For example:
\begin{smalltt}    
    ...
    [| x := 3 ; y := z |] \{\}   -|>
     (by assign.2 )
     (by seq.1 )
    [| skip ; y := z |] x |-> 3 & \{\}   -|>
     (by seq.2 )
    [| y := z |] x |-> 3 & \{\}   -|>
\end{smalltt}
\subsubsection*{Tasks} 

\begin{enumerate}
\item 
Encode the rules for addition so that they prescribe a left-to-right
order of evaluation.
\item 
Define the following ``driver'' relation
\begin{smalltt}
    #notfix run
    run Program :- [| Program |] \{\} -|>* Config'
\end{smalltt}
and give evidence that the encoding of the rules you gave in part 1 
is appropriate  by
running the following predicates.
\begin{smalltt}
        prob1  :- run x:=3 .
        prob2  :- run x:=3; y:=x .
        prob2a :- run x:=3; y:=z .
        prob3  :- run x:=3; y:=4; z:=x .
        prob4  :- run x:=3; y:=4; z:=x+y .
\end{smalltt}
\item 
Change the rules for addition so that they no longer prescribe left-to-right
evaluation, and try running {\tt prob4} again. What do you notice?


\item 
Encode the rules for {\tt after} and gather evidence of their
appropriateness by running (with the order-prescriptive rules for
addition back in place)
\begin{smalltt}
        prob5 :- run x:=3; y:=4; z:=(x after x:=x+1) .
\end{smalltt}
{\bf Hint:} You will  need to change a few of the rules so that they permit
the evaluation of expressions with side-effects. Make a
note of what happens before you change them.

\item Optional tasks (independent of each other)
\begin{enumerate}
\item ({\it for the moderately ambitious\/})
Encode the rules for conditionals, and gather evidence of
their appropriateness. You will need to add relational operators
to the language. If you are successful in this then encode the iteration rules.

{\bf Hint:}  It is wisest to represent these constructs directly ({\it i.e.} in their
abstract syntax forms, as exemplified below), rather than trying to 
find  syntactic declarations for \LOGPRO which make it possible to write them in
the (admittedly more readable) concrete forms given in the notes.
\begin{smalltt}
        if(\(condition\), \(TrueProgram\), \(FalseProgram\))
        while(\(condition\), \(LoopBody\))
\end{smalltt}

\item ({\it for the somewhat persistent\/})
Try running the following predicate
with non-order-prescriptive rules for addition in place
\begin{smalltt}
        prob6 :- run x:=3; y:=4; z:=(x after x:=x+1)+x.
\end{smalltt}
Can you explain what is happening?

Now, if you have time, {\it and you are more than just somewhat persistent\/} try 
the following with the same rules in place
\begin{smalltt}
        prob7 :- run x:=3; y:=4; z:=(x after x:=y)+(y after y:=x).
\end{smalltt}
Can you explain what is happening?

\end{enumerate}
\end{enumerate}
%---------------------------------------------------------------------------
\newpage
\appendix
\section{The \LOGPRO Manual (for version \LOGPROVERSION)}\label{Manual}
\newcommand{\sect}[1]{\subsection{#1}}
\newcommand{\subsect}[1]{\subsubsection{#1}}
%\RCS $Id: manual.tex,v 1.3 2000/02/07 15:08:00 sufrin Exp sufrin $
%
%
%
\def\rmbox#1{\mbox{{\rm #1}}}
%
%
%
\sect{Running \LOGPRO}
The \LOGPRO interpreters are available in the Laboratory filestore as
\begin{smalltt}
        /ecslab/sufrin/logpro/bin/logpro
        /ecslab/sufrin/logpro/bin/x86-solaris/logpro
        /ecslab/sufrin/logpro/bin/sparc-solaris/logpro
        /ecslab/sufrin/logpro/bin/x86-linux/logpro
\end{smalltt}

The first of these is a bytecode interpreter, and can be used anywhere there is an
installation of ocaml 2.02. The others are binary executables which can
be used on a machine with the appropriate architecture and operating system.\footnote{The Linux 
version was compiled and runs under a completely standard RedHat 6.1 distribution. I don't
know of any reason why it might not run under other Linux distributions.}


Once the appropriate path is added to your PATH variable, the \LOGPRO interpreter is invoked with the command
\begin{smalltt}
        logpro [path path ... path]
\end{smalltt}

After the  commands in the specified files have been read,\footnote{For each
$path$, \LOGPRO reads the file $path\mathtt{.lp}$ if it exists, and otherwise
reads the file $path$.}
the  intepreter enters interactive mode, initially prompting for
each command with \verb|--|,\footnote{If 
the interpreter requires more input in response to 
its prompt -- perhaps because a query or directive has not yet been terminated properly, then
it will prompt again with its secondary prompt: {\tt >>}} and responding to queries by
displaying the substitutions which satisfy them\footnote{Or ``{\tt Yes}'' if there are no
variables in the query} in sequence. After each substitution is displayed,
the interpreter prompts with a ``\verb|?|'': responding to
this with ``\verb|.|'' terminates the search for substitutions;
responding with a newline causes the search to continue. At any
stage during the evaluation of a query, it can be interrupted
by typing the appropriate interrupt character for the operating
system on which it is being run.





\sect{Commands}
A \LOGPRO command is either a \emph{clause} taking the form
\begin{smalltt}
        \(formula\) :- \(formula\), \(formula\), \(\cdots\) \(formula\).
\end{smalltt}
or a \emph{query} taking one of the (equivalent) forms
\begin{smalltt}
        :- \(formula\), \(formula\), \(\cdots\) \(formula\).
           \(formula\), \(formula\), \(\cdots\) \(formula\)?
\end{smalltt}
or a \emph{directive} taking one of the forms in Table \ref{Directives}.

If there are no formul\ae\xspace to the right of the
entailment symbol \verb.:-.  in a clause then that symbol
may be omitted.


\begin{center}
\begin{table}[hpb]
\def\cmd#1{{\tt \##1}}
\def\Q{{\tt"}}
\begin{tabular}{|ll|l|}\hline
    Command && Effect\\\hline\hline
    \cmd{use}      &\(path\)          &\rmbox{Read the file specified by the given path (no more than once per session)}
    \\\cmd{width}  &\(number\)        &\rmbox{Set the output width to $number$ columns}
    \\\cmd{infix0} &\Q\(symbol\)\Q    &\rmbox{Declare the  symbol to be a right-associative infix operator, priority 0}             
    \\\cmd{infix1} &\Q\(symbol\)\Q    &\rmbox{Declare the  symbol to be a right-associative infix operator, priority 1}  
    \\...          &                  &
    \\\cmd{infix7} &\Q\(symbol\)\Q    &\rmbox{Declare the  symbol to be a right-associative infix operator, priority 7}  
    \\\cmd{infix8} &\Q\(symbol\)\Q    &\rmbox{Declare the  symbol to be a \emph{left-associative} infix operator, priority 8} 
    \\\cmd{prefix} &\Q\(symbol\)\Q    &\rmbox{Declare the  symbol to be a prefix operator with maximal priority} 
    \\\cmd{notfix} &\Q\(symbol\)\Q    &\rmbox{Declare the  symbol to be a prefix operator with minimal priority} 
    \\\cmd{trace}  &on                &\rmbox{Trace the invocation of each predicate}
    \\\cmd{trace}  &all               &\rmbox{As above and show the complete current substitution at each invocation}
    \\\cmd{trace}  &off               &\rmbox{Stop tracing}
    \\\cmd{check}  &on \(or\) off     &\rmbox{Enable or disable the occurs check in unification}
    \\\cmd{count}  &on \(or\) off     &\rmbox{Enable or disable the printing of inference count with answers}
    \\\hline
\end{tabular}
\caption{Directives}
\label{Directives}
\end{table}
\end{center}

\sect{Notation}
Any text appearing between a \verb"/*" and the  subsequent
\verb"*/" is treated as a comment and ignored, as is any text
appearing between a \verb"--" and the subsequent newline.


A \emph{word} is any consecutive sequence of letters, digits, underscores or primes.

A \emph{symbol} is any consecutive sequence of one or more of the  characters
``\verb.!^%$@&/\;=><.'',  ``\verb.:+-*/~|.''  

The symbol \verb":-" is the \emph{entailment} symbol. 
Any other symbol or word may be declared to be an infix, prefix, or notfix operator using
one of the directives outlined in Table \ref{Directives}. Unless declared otherwise by such a 
directive, all symbols behave as left associative infix operators of priority 8.

A \emph{term} has one of the forms described below, whilst a {\em formula} has any of these 
but the first three.
\begin{enumerate}
\item  a (possibly negated) \emph{integer}. For example:
\begin{smalltt}
        34567
        -32
\end{smalltt}
\item  a \emph{double-quoted string}.
\item  a \emph{variable} -- which is a word beginning with an uppercase letter. For example:
\begin{smalltt}
        X
        Rumpelstiltskin
        State_Space
\end{smalltt}
 
\item  an \emph{atom} --  which is a word beginning with a lowercase letter. For example:
\begin{smalltt}
        it_is_raining
        david
        pruneSquallor
\end{smalltt}
\item  a \emph{prefix composite} -- an atom, followed by a parenthesized, 
        comma-separated, list of \emph{term}s.  
For example:
\begin{smalltt}
        grandparent(X, Y)
        connected(From, To, Via)
        simply(purple)
        negative(2)
\end{smalltt}
\item an \emph{infix composite} -- a pair of terms separated by an infix operator. 
For example, after the directive {\tt \#infix3 "divides"}
\begin{smalltt}
        Var + Const
        3 divides 62
\end{smalltt}
\item a \emph{leftfix composite} -- a term followed by a bracketed term.
        This is regarded as syntactic sugar 
        for a prefix composite with functor {\tt{[]}}.
For example:
\begin{smalltt}
        f[Arg] \rmbox{is sugar for} [](f, Arg)
        g[x][y] \rmbox{is sugar for} []([](g, x), y)
\end{smalltt}
\item a \emph{rightfix composite} -- a term bracketed by the ``fat
brackets'' \tt{[|} and \tt{|]}. This is regarded as syntactic sugar 
        for a prefix composite with functor {\tt{[||]}}.
For example:
\begin{smalltt}
        [|I|] S \rmbox{is sugar for} [||](I, S)
        [|I|][|J|]S \rmbox{is sugar for} [||](I, [||](J, S))
\end{smalltt}

Leftfix composites have higher priority than rightfix composites. For example:
\begin{smalltt}
        [|I|]J[S] \rmbox{is sugar for} [||](I, [](J, S))
\end{smalltt}
\item an \emph{outfix composite} -- a curly-bracketed, 
        comma-separated, possibly-empty, list of \emph{term}s. 
        This is regarded as syntactic sugar 
        for a prefix composite with functor {\tt{\{\}}}.
For example:
\begin{smalltt}
        \{\} \rmbox{is sugar for} \{\}()
        \{foo, bar\} \rmbox{is sugar for} \{\}(foo, bar)
\end{smalltt}

\item  a 
\emph{notfix operator} followed by a \emph{term}. The only built-in 
notfix operator is {\tt not}, but others may be declared using the {\tt{\#notfix}}
directive (Table \ref{Directives}). \emph{Notfix operator}s have less priority than 
any other forms of operator. 
For example
\begin{smalltt}
        not male(X)
        not append(X, Y, Z)
        not not contradiction(X)
\end{smalltt}
With directives {\tt \#notfix "let"}, and {\tt \#infix3 "in"}
\begin{smalltt}
        let x=3 in x+2 
\end{smalltt}
yields the formula whose structure is
\begin{smalltt}
        let(in((=)(x, 3), (+)(x, 2)))
\end{smalltt}


\item a 
\emph{prefix operator} followed by a \emph{term}. There are no
built-in prefix operators, but they may be declared using the {\tt{\#prefix}}
directive (Table \ref{Directives}). \emph{Prefix operator}s have higher priority than 
infix operators. 
For example, after the directives {\tt \#prefix "factor"} and
{\tt \#infix3 "or"}
\begin{smalltt}
        factor Y or magic(Y, Z)
\end{smalltt}
\end{enumerate}

\sect{Built-In Relations}
The relations and symbols built-in to \LOGPRO are described in Table \ref{Builtin}.
\begin{center}
\begin{table}[hpb]
\def\|#1|#2|{{\tt #1}&$#2$&}
\begin{tabular}{|l|l|l|}\hline
Relation        &Arguments  &Effect (Restriction)\\\hline\hline
\|!             |           |the cut symbol 
\\
\|fail          |           |always fails 
\\
\|sum           |(A,B,C)    |$C$ is the sum of $A$ and $B$. (No more than one variable) 
\\
\|prod          |(A,B,C)    |$C$ is the product of $A$ and $B$. (No more than one variable) 
\\
\|succ          |(A, B)     |$B$ is the natural number successor to $A$. (No more than one variable)
\\
\|>=            |(A, B)     |$A \geq B$. (No variables).
\\
\|>             |(A, B)     |$A > B$. (No variables).
\\
\|<=            |(A, B)     |$A \leq B$. (No variables).
\\
\|<             |(A, B)     |$A < B$. (No variables).
\\
\|num           |(A)        |$A$ is an integer.
\\
\|str           |(A)        |$A$ is a string.
\\
\|atom          |(A)        |$A$ is an atom.
\\
\|var           |(A)        |$A$ is an uninstantiated variable.
\\
\|nonvar        |(A)        |$A$ is not an uninstantiated variable.
\\
\|toString      |(A,B)      |$B$ is the string representing the term $A$.
\\
\|toFormula     |(A,B)      |$B$ is the formula represented by the string $A$.
\\
\|functor       |(A,OP,B)   |$A$ is the formula with principal connective $OP$ and list of arguments $B$.
\\
\|call          |(A, B, ...)|The formulae $A, B, ...$ are evaluated in sequence.
\\
\|len           |(A,B)      |$B$ is the length of the string $A$.
\\
\|cat           |(A,B,C)    |$C$ is the catenation of the strings $A$ and $B$ (No more than one variable).
\\
\|hd            |(A,B)      |$B$ is a string consisting of the first character of the string $A$ (ditto).
\\
\|tl            |(A,B)      |$B$ consists of all but the first character of the string $A$ (ditto).
\\
\|ascii         |(A,B)      |$B$ is the ascii code of the first character of the string $A$ (ditto).
\\
\|read          |(A)        |The string $A$ is the next line read from the terminal.
\\
\|show          |(A, B, ...)|The terms $A, B, ...$ are printed on the terminal followed by a newline.
\\
\|write         |(A, B, ...)|The terms $A, B, ...$ are printed on the terminal.
\\
\hline
\multicolumn{3}{|c|}{The following relations manipulate the built-in updateable mapping from terms to terms}
\\
\hline
\|map\_Lookup|(A, B)|$B$ is the current value of the mapping at key $A$. 
\\
\|map\_Add|(A, B)|The value of the mapping at key $A$ is overlaid by $B$. 
\\
\|map\_Enter|(A, B)|The value of the mapping at key $A$ becomes $B$; previous values removed. 
\\
\|map\_Remove|(A)|Remove the topmost value of the mapping at key $A$. 
\\
\|map\_All|(A, B)|$B$ is the list of values of the mapping at key $A$. 
\\
\hline
\end{tabular}
\caption{Built-in Relations}
\label{Builtin}
\end{table}
\end{center}

\sect{Changes}

\begin{enumerate}
\item[V1.44]  
\begin{enumerate}
\item The syntax of terms has been extended to accomodate ``fat brackets''. 
\item Different primary and secondary prompts ({\tt --} and {\tt >>} respectively)
\end{enumerate}\end{enumerate}
















%---------------------------------------------------------------------------
\clearpage
\pagenumbering{roman}
\newpage
\tableofcontents
%\RCSSignature

Typeset on \today

\end{document}












