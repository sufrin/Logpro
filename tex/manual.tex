%\RCS $Id: manual.tex,v 1.3 2000/02/07 15:08:00 sufrin Exp sufrin $
%
%
%
\def\rmbox#1{\mbox{{\rm #1}}}
%
%
%
\sect{Running \LOGPRO}
The \LOGPRO interpreters are available in the Laboratory filestore as
\begin{smalltt}
        /ecslab/sufrin/logpro/bin/logpro
        /ecslab/sufrin/logpro/bin/x86-solaris/logpro
        /ecslab/sufrin/logpro/bin/sparc-solaris/logpro
        /ecslab/sufrin/logpro/bin/x86-linux/logpro
\end{smalltt}

The first of these is a bytecode interpreter, and can be used anywhere there is an
installation of ocaml 2.02. The others are binary executables which can
be used on a machine with the appropriate architecture and operating system.\footnote{The Linux 
version was compiled and runs under a completely standard RedHat 6.1 distribution. I don't
know of any reason why it might not run under other Linux distributions.}


Once the appropriate path is added to your PATH variable, the \LOGPRO interpreter is invoked with the command
\begin{smalltt}
        logpro [path path ... path]
\end{smalltt}

After the  commands in the specified files have been read,\footnote{For each
$path$, \LOGPRO reads the file $path\mathtt{.lp}$ if it exists, and otherwise
reads the file $path$.}
the  intepreter enters interactive mode, initially prompting for
each command with \verb|--|,\footnote{If 
the interpreter requires more input in response to 
its prompt -- perhaps because a query or directive has not yet been terminated properly, then
it will prompt again with its secondary prompt: {\tt >>}} and responding to queries by
displaying the substitutions which satisfy them\footnote{Or ``{\tt Yes}'' if there are no
variables in the query} in sequence. After each substitution is displayed,
the interpreter prompts with a ``\verb|?|'': responding to
this with ``\verb|.|'' terminates the search for substitutions;
responding with a newline causes the search to continue. At any
stage during the evaluation of a query, it can be interrupted
by typing the appropriate interrupt character for the operating
system on which it is being run.





\sect{Commands}
A \LOGPRO command is either a \emph{clause} taking the form
\begin{smalltt}
        \(formula\) :- \(formula\), \(formula\), \(\cdots\) \(formula\).
\end{smalltt}
or a \emph{query} taking one of the (equivalent) forms
\begin{smalltt}
        :- \(formula\), \(formula\), \(\cdots\) \(formula\).
           \(formula\), \(formula\), \(\cdots\) \(formula\)?
\end{smalltt}
or a \emph{directive} taking one of the forms in Table \ref{Directives}.

If there are no formul\ae\xspace to the right of the
entailment symbol \verb.:-.  in a clause then that symbol
may be omitted.


\begin{center}
\begin{table}[hpb]
\def\cmd#1{{\tt \##1}}
\def\Q{{\tt"}}
\begin{tabular}{|ll|l|}\hline
    Command && Effect\\\hline\hline
    \cmd{use}      &\(path\)          &\rmbox{Read the file specified by the given path (no more than once per session)}
    \\\cmd{width}  &\(number\)        &\rmbox{Set the output width to $number$ columns}
    \\\cmd{infix0} &\Q\(symbol\)\Q    &\rmbox{Declare the  symbol to be a right-associative infix operator, priority 0}             
    \\\cmd{infix1} &\Q\(symbol\)\Q    &\rmbox{Declare the  symbol to be a right-associative infix operator, priority 1}  
    \\...          &                  &
    \\\cmd{infix7} &\Q\(symbol\)\Q    &\rmbox{Declare the  symbol to be a right-associative infix operator, priority 7}  
    \\\cmd{infix8} &\Q\(symbol\)\Q    &\rmbox{Declare the  symbol to be a \emph{left-associative} infix operator, priority 8} 
    \\\cmd{prefix} &\Q\(symbol\)\Q    &\rmbox{Declare the  symbol to be a prefix operator with maximal priority} 
    \\\cmd{notfix} &\Q\(symbol\)\Q    &\rmbox{Declare the  symbol to be a prefix operator with minimal priority} 
    \\\cmd{trace}  &on                &\rmbox{Trace the invocation of each predicate}
    \\\cmd{trace}  &all               &\rmbox{As above and show the complete current substitution at each invocation}
    \\\cmd{trace}  &off               &\rmbox{Stop tracing}
    \\\cmd{check}  &on \(or\) off     &\rmbox{Enable or disable the occurs check in unification}
    \\\cmd{count}  &on \(or\) off     &\rmbox{Enable or disable the printing of inference count with answers}
    \\\hline
\end{tabular}
\caption{Directives}
\label{Directives}
\end{table}
\end{center}

\sect{Notation}
Any text appearing between a \verb"/*" and the  subsequent
\verb"*/" is treated as a comment and ignored, as is any text
appearing between a \verb"--" and the subsequent newline.


A \emph{word} is any consecutive sequence of letters, digits, underscores or primes.

A \emph{symbol} is any consecutive sequence of one or more of the  characters
``\verb.!^%$@&/\;=><.'',  ``\verb.:+-*/~|.''  

The symbol \verb":-" is the \emph{entailment} symbol. 
Any other symbol or word may be declared to be an infix, prefix, or notfix operator using
one of the directives outlined in Table \ref{Directives}. Unless declared otherwise by such a 
directive, all symbols behave as left associative infix operators of priority 8.

A \emph{term} has one of the forms described below, whilst a {\em formula} has any of these 
but the first three.
\begin{enumerate}
\item  a (possibly negated) \emph{integer}. For example:
\begin{smalltt}
        34567
        -32
\end{smalltt}
\item  a \emph{double-quoted string}.
\item  a \emph{variable} -- which is a word beginning with an uppercase letter. For example:
\begin{smalltt}
        X
        Rumpelstiltskin
        State_Space
\end{smalltt}
 
\item  an \emph{atom} --  which is a word beginning with a lowercase letter. For example:
\begin{smalltt}
        it_is_raining
        david
        pruneSquallor
\end{smalltt}
\item  a \emph{prefix composite} -- an atom, followed by a parenthesized, 
        comma-separated, list of \emph{term}s.  
For example:
\begin{smalltt}
        grandparent(X, Y)
        connected(From, To, Via)
        simply(purple)
        negative(2)
\end{smalltt}
\item an \emph{infix composite} -- a pair of terms separated by an infix operator. 
For example, after the directive {\tt \#infix3 "divides"}
\begin{smalltt}
        Var + Const
        3 divides 62
\end{smalltt}
\item a \emph{leftfix composite} -- a term followed by a bracketed term.
        This is regarded as syntactic sugar 
        for a prefix composite with functor {\tt{[]}}.
For example:
\begin{smalltt}
        f[Arg] \rmbox{is sugar for} [](f, Arg)
        g[x][y] \rmbox{is sugar for} []([](g, x), y)
\end{smalltt}
\item a \emph{rightfix composite} -- a term bracketed by the ``fat
brackets'' \tt{[|} and \tt{|]}. This is regarded as syntactic sugar 
        for a prefix composite with functor {\tt{[||]}}.
For example:
\begin{smalltt}
        [|I|] S \rmbox{is sugar for} [||](I, S)
        [|I|][|J|]S \rmbox{is sugar for} [||](I, [||](J, S))
\end{smalltt}

Leftfix composites have higher priority than rightfix composites. For example:
\begin{smalltt}
        [|I|]J[S] \rmbox{is sugar for} [||](I, [](J, S))
\end{smalltt}
\item an \emph{outfix composite} -- a curly-bracketed, 
        comma-separated, possibly-empty, list of \emph{term}s. 
        This is regarded as syntactic sugar 
        for a prefix composite with functor {\tt{\{\}}}.
For example:
\begin{smalltt}
        \{\} \rmbox{is sugar for} \{\}()
        \{foo, bar\} \rmbox{is sugar for} \{\}(foo, bar)
\end{smalltt}

\item  a 
\emph{notfix operator} followed by a \emph{term}. The only built-in 
notfix operator is {\tt not}, but others may be declared using the {\tt{\#notfix}}
directive (Table \ref{Directives}). \emph{Notfix operator}s have less priority than 
any other forms of operator. 
For example
\begin{smalltt}
        not male(X)
        not append(X, Y, Z)
        not not contradiction(X)
\end{smalltt}
With directives {\tt \#notfix "let"}, and {\tt \#infix3 "in"}
\begin{smalltt}
        let x=3 in x+2 
\end{smalltt}
yields the formula whose structure is
\begin{smalltt}
        let(in((=)(x, 3), (+)(x, 2)))
\end{smalltt}


\item a 
\emph{prefix operator} followed by a \emph{term}. There are no
built-in prefix operators, but they may be declared using the {\tt{\#prefix}}
directive (Table \ref{Directives}). \emph{Prefix operator}s have higher priority than 
infix operators. 
For example, after the directives {\tt \#prefix "factor"} and
{\tt \#infix3 "or"}
\begin{smalltt}
        factor Y or magic(Y, Z)
\end{smalltt}
\end{enumerate}

\sect{Built-In Relations}
The relations and symbols built-in to \LOGPRO are described in Table \ref{Builtin}.
\begin{center}
\begin{table}[hpb]
\def\|#1|#2|{{\tt #1}&$#2$&}
\begin{tabular}{|l|l|l|}\hline
Relation        &Arguments  &Effect (Restriction)\\\hline\hline
\|!             |           |the cut symbol 
\\
\|fail          |           |always fails 
\\
\|sum           |(A,B,C)    |$C$ is the sum of $A$ and $B$. (No more than one variable) 
\\
\|prod          |(A,B,C)    |$C$ is the product of $A$ and $B$. (No more than one variable) 
\\
\|succ          |(A, B)     |$B$ is the natural number successor to $A$. (No more than one variable)
\\
\|>=            |(A, B)     |$A \geq B$. (No variables).
\\
\|>             |(A, B)     |$A > B$. (No variables).
\\
\|<=            |(A, B)     |$A \leq B$. (No variables).
\\
\|<             |(A, B)     |$A < B$. (No variables).
\\
\|num           |(A)        |$A$ is an integer.
\\
\|str           |(A)        |$A$ is a string.
\\
\|atom          |(A)        |$A$ is an atom.
\\
\|var           |(A)        |$A$ is an uninstantiated variable.
\\
\|nonvar        |(A)        |$A$ is not an uninstantiated variable.
\\
\|toString      |(A,B)      |$B$ is the string representing the term $A$.
\\
\|toFormula     |(A,B)      |$B$ is the formula represented by the string $A$.
\\
\|functor       |(A,OP,B)   |$A$ is the formula with principal connective $OP$ and list of arguments $B$.
\\
\|call          |(A, B, ...)|The formulae $A, B, ...$ are evaluated in sequence.
\\
\|len           |(A,B)      |$B$ is the length of the string $A$.
\\
\|cat           |(A,B,C)    |$C$ is the catenation of the strings $A$ and $B$ (No more than one variable).
\\
\|hd            |(A,B)      |$B$ is a string consisting of the first character of the string $A$ (ditto).
\\
\|tl            |(A,B)      |$B$ consists of all but the first character of the string $A$ (ditto).
\\
\|ascii         |(A,B)      |$B$ is the ascii code of the first character of the string $A$ (ditto).
\\
\|read          |(A)        |The string $A$ is the next line read from the terminal.
\\
\|show          |(A, B, ...)|The terms $A, B, ...$ are printed on the terminal followed by a newline.
\\
\|write         |(A, B, ...)|The terms $A, B, ...$ are printed on the terminal.
\\
\hline
\multicolumn{3}{|c|}{The following relations manipulate the built-in updateable mapping from terms to terms}
\\
\hline
\|map\_Lookup|(A, B)|$B$ is the current value of the mapping at key $A$. 
\\
\|map\_Add|(A, B)|The value of the mapping at key $A$ is overlaid by $B$. 
\\
\|map\_Enter|(A, B)|The value of the mapping at key $A$ becomes $B$; previous values removed. 
\\
\|map\_Remove|(A)|Remove the topmost value of the mapping at key $A$. 
\\
\|map\_All|(A, B)|$B$ is the list of values of the mapping at key $A$. 
\\
\hline
\end{tabular}
\caption{Built-in Relations}
\label{Builtin}
\end{table}
\end{center}

\sect{Changes}

\begin{enumerate}
\item[V1.44]  
\begin{enumerate}
\item The syntax of terms has been extended to accomodate ``fat brackets''. 
\item Different primary and secondary prompts ({\tt --} and {\tt >>} respectively)
\end{enumerate}\end{enumerate}















